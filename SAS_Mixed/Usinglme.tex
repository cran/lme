%%% $Id: Usinglme.tex,v 1.4 1998/04/01 23:19:49 bates Exp $
\documentclass[]{article}
\NeedsTeXFormat{LaTeX2e}[1995/06/01]
\usepackage[dcu,dcucite]{harvard}
\usepackage{times}
\usepackage{amsbsy}
\usepackage{amsmath}
\usepackage{graphicx}
\usepackage{alltt}
\graphicspath{{figs/}}
\DeclareGraphicsExtensions{.epsi.gz,.epsi,.eps.gz,.eps,.ps,.ps.gz}
\DeclareGraphicsRule{*}{eps}{*}{}
\DeclareGraphicsRule{.epsi.gz}{eps}{.bb}{`zcat #1}
\DeclareGraphicsRule{.eps.gz}{eps}{.bb}{`zcat #1}
\DeclareGraphicsRule{.ps.gz}{eps}{.bb}{`zcat #1}
\setkeys{Gin}{width=\textwidth}
%%% $Id: defs.tex,v 1.2 1998/04/01 23:19:49 bates Exp $
\newcommand{\abs}{\operatorname{abs}}
\newcommand{\tr}{\operatorname{tr}}
\newcommand{\bm}[1]{{\boldsymbol {#1}}}
\newcommand{\Ti}[1]{\centerline{\textbf{\large\blue #1}}}
\newcommand{\mg}[1]{{\magenta #1}}
\newcommand{\gaus}{{\cal N}}
\newcommand{\Cf}[1]{\textsf{#1}}
\newcommand{\E}[1]{\mbox{E}\left(#1\right)}
\newcommand{\bA}{{\bm A}}
\newcommand{\bB}{{\bm B}}
\newcommand{\bD}{{\bm D}}
\newcommand{\bDelta}{{\bm\Delta}}
\newcommand{\bI}{{\bm I}}
\newcommand{\bLambda}{\bm\Lambda}
\newcommand{\bSigma}{\bm\Sigma}
\newcommand{\bQ}{{\bm Q}}
\newcommand{\bR}{{\bm R}}
\newcommand{\bS}{{\bm S}}
\newcommand{\bT}{{\bm T}}
\newcommand{\bU}{{\bm U}}
\newcommand{\bX}{{\bm X}}
\newcommand{\bZ}{{\bm Z}}
\newcommand{\ba}{{\bm a}}
\newcommand{\balpha}{\bm\alpha}
\newcommand{\bbeta}{\bm\beta}
\newcommand{\bbf}{\bm f}
\newcommand{\bb}{{\bm b}}
\newcommand{\bc}{{\bm c}}
\newcommand{\bd}{{\bm d}}
\newcommand{\bdelta}{{\bm \delta}}
\newcommand{\beps}{\bm\epsilon}
\newcommand{\bphi}{\bm\phi}
\newcommand{\brho}{\bm\rho}
\newcommand{\btau}{\bm{\tau}}
\newcommand{\btheta}{\bm{\theta}}
\newcommand{\bv}{{\bm v}}
\newcommand{\bx}{{\bm x}}
\newcommand{\by}{{\bm y}}
\newcommand{\bz}{{\bm z}}
\newcommand{\bzer}{{\bm 0}}
\newcommand{\simind}{\mbox{$\stackrel{\rm ind}{\sim}$}}
\newcommand{\s}{\textsf{S}}
\newcommand{\R}{\textsf{R}}
\newcommand{\Splus}{\textsf{S-PLUS}}
\newcommand{\Co}[1]{{\small\texttt{#1}}}
\newcommand{\trans}{^\prime{}}
%%% Local Variables: 
%%% mode: latex
%%% TeX-master: "all"
%%% End: 

\newenvironment{Example}%
{\begin{list}{}{\setlength{\leftmargin}{.5\parindent}}%
  \item\begin{alltt}\small}%
    {\end{alltt}\end{list}}
\pagenumbering{arabic}
\begin{document}
\title{\textbf{\textsf{lme} for \textsf{SAS PROC MIXED} Users}}
\author{Douglas M.~Bates\smallskip\\
  \textsf{\normalsize Department of Statistics}\\[-0.3ex]
  \textsf{\normalsize University of Wisconsin -- Madison}
  \thanks{This research was supported by the National Science
    Foundation through grant DMS-9704349.}
  \and  Jos\'e C.~Pinheiro\smallskip\\
  \textsf{\normalsize Bell Laboratories}\\[-0.3ex]
  \textsf{\normalsize Lucent Technologies}}
\date{}
\maketitle

\section{Introduction}
\label{sec:intro}

The \Co{lme} function from the \Co{nlme} library for \Splus{} or the
\Co{lme} library for \R{} is used to fit linear mixed-effects models.
It is similar in scope to the \textsf{SAS} procedure \Co{PROC MIXED}
described in \citeasnoun{litt:mill:stro:wolf:1996}.

A file on the SAS Institute web site (\textsf{http://www.sas.com})
contains all the data sets in the book and all the SAS programs used
in \citeasnoun{litt:mill:stro:wolf:1996}.  We have converted the data
sets from the tabular representation used for SAS to the
\Co{groupedData} objects used by \Co{lme}.  To help users familiar
with \Co{SAS PROC MIXED} get up to speed with \Co{lme} more quickly,
we provide transcripts of some \Co{lme} analyses paralleling the
\Co{SAS PROC MIXED} analyses in \citeasnoun{litt:mill:stro:wolf:1996}.

In this paper we highlight some of the similarities and differences of
\Co{lme} analysis and \Co{SAS PROC MIXED} analysis.

\section{Similarities between lme and SAS PROC MIXED}
\label{sec:similarities}

Both \Co{SAS PROC MIXED} and \Co{lme} can fit linear mixed-effects
models expressed in the Laird-Ware formulation.  For a single level of
grouping \citeasnoun{lair:ware:1982} write the $n_i\/$-dimensional
response vector $\by_i$ for the $i\/$th experimental unit as
\begin{gather}
  \label{eqn:oneLevel}
  \by_i = \bX_i \bbeta + \bZ_i \bb_i + \beps_i,\quad i=1,\dots,M\\
  \bb_i\sim\gaus(\bzer,\bSigma),
  \quad\beps_i\sim\gaus(\bzer,\sigma^2 \bI)\notag
\end{gather}
where $\bbeta$ is the $p$-dimensional vector of \emph{fixed effects},
$\bb_i$ is the $q$-dimensional vector of \emph{random effects},
$\bX_i$ (of size $n_i\times p$) and $\bZ_i$ (of size $n_i\times q$)
are known fixed-effects and random-effects regressor matrices, and
$\beps_i$ is the $n_i\/$-dimensional \emph{within-group error} vector
with a spherical Gaussian distribution.  The assumption
$\mathrm{Var}(\beps_i)=\sigma^2\bI$ can be relaxed using additional
arguments in the model fitting.

The basic specification of the model requires a linear model
expression for the fixed effects and a linear model expression for the 
random effects.  In \Co{SAS PROC MIXED} the fixed-effects part is
specified in the \Co{model} statement and the random-effects
part in the \Co{random} statement.  In \Co{lme} the
arguments are called \Co{fixed} and \Co{random}.

Both \Co{SAS PROC MIXED} and \Co{lme} allow a mixed-effects model to
be fit by maximum likelihood (\Co{method = ml} in SAS) or by maximum
residual likelihood, sometimes also called restricted maximum
likelihood or \textsf{REML}.  This is the default criterion in \Co{SAS
  PROC MIXED}.  The default criterion in \Co{lme} is maximum
likelihood.  To get \textsf{REML} estimates in \Co{lme}, set the
optional argument \Co{REML=TRUE}.


\section{Important differences}
\label{sec:differences}

One of the most important differences has just been stated but is
worth repeating.  SAS defaults to \textsf{REML} fits; \Co{lme}
defaults to maximum likelihood fits.

The output from \Co{PROC MIXED} typically includes values of the
Akaike Information Criterion (\textsf{AIC}) and Schwartz's Bayesian
Criterion (\textsf{SBC}).  These are used to compare different models
fit to the same data.  The output of the \Co{summary} function applied
to the object created by \Co{lme} also produces values of \textsf{AIC}
and \textsf{BIC} but the definitions used in \Co{PROC MIXED} and in
\Co{lme} are different.  In \Co{lme} the definitions are such that
``smaller is better''.  In \Co{PROC MIXED} the definitions are such
that ``bigger is better''.

When models are fit by \textsf{REML}, the values of \textsf{AIC},
\textsf{SBC} (or \textsf{BIC}) and the log-likelihood can only be
compared between models with exactly the same fixed-effects structure.
When models are fit by maximum likelihood these criteria can be
compared between any models fit to the same data.  That is, these
quality-of-fit criteria can be used to evaluate different
fixed-effects specifications or different random-effects
specifications or different specifications of both fixed effects and
random effects.  The greater flexibility of model comparisons when
using maximum likelihood is the reason that this is the default
criterion in \Co{lme}.

We encourage developing and testing the model using likelihood ratio
tests or the \textsf{AIC} and \textsf{BIC} criteria.  Once a form
for both the random effects and the fixed effects has been determined,
the model can be refit with \Co{REML = TRUE} if the restricted
estimates of the variance components are desired.

\section{Data manipulation}
\label{sec:data}

Both \Co{PROC MIXED} and \Co{lme} work with data in a tabular form
with one row per observation.  There are, however, important
differences in the internal representations of variables in the data.

In \textsf{SAS} a qualitative factor can be stored either as numerical
values or alphanumeric labels.  When a factor stored as numerical
values is used in \Co{PROC MIXED} it is listed in the \Co{class}
statement to indicate that it is a factor.  In \s{} this information
is stored with the data itself by converting the variable to a factor
when it is first stored.  If the factor represents an ordered set of
levels, it should be converted to an \Co{ordered} factor.

For example the SAS code
\begin{Example}
data animal;
 input trait animal y;
 datalines;
1 1 6
1 2 8
1 3 7
2 1 9
2 2 5
2 3 .
;
\end{Example}
would require that the \Co{trait} and \Co{animal} variables be
specified in a class statement in any model that is fit.

In \s{} these data could be read from a file, say \texttt{animal.dat},
and converted to factors by
\begin{Example}
S> animal <- read.table( "animal.dat", header = TRUE )
S> class( animal )
[1] "data.frame"
S> animal$trait <- as.factor( animal$trait )
S> animal$animal <- as.factor( animal$animal )
\end{Example}
In general it is a good idea to check the types of variables in a data 
frame before working with it.  One way of doing this is to apply
the function \textsf{data.class} to each variable in turn using the
\Co{sapply} function.
\begin{Example}
S> sapply( animal, data.class )
    trait    animal         y 
 "factor"  "factor" "numeric" 
\end{Example}

To make specification of models in \Co{lme} easier and to make graphic
presentations more informative, we recommend converting from a
\Co{data.frame} object to a \Co{groupedData} object.  This class of
objects contains a formula specifying the response, the primary
covariate (if there is one) and the grouping factor or factors.  The
data sets from \citeasnoun{litt:mill:stro:wolf:1996} have been
converted to \Co{groupedData} objects in this directory.

\subsection{Unique levels of factors}
\label{sec:nested}

Designs with nested grouping factors are indicated differently in the
two languages.  An example of such an experimental design is the
semiconductor experiment described in section 2.2 of
\citeasnoun{litt:mill:stro:wolf:1996} where twelve wafers are 
assigned to four experimental treatments with three wafers per
treatment.  The levels for the wafer factor are \Co{1}, \Co{2}, and
\Co{3} but the wafer factor is only meaningful within the same level
of the treatment factor, \Co{et}.  There is nothing associating wafer
\Co{1} of the third treatment group with wafer \Co{1} of the first
treatment group.

In \Co{SAS} this nesting of factors is denoted by \Co{wafer(et)}.  In
\s{} the nesting is written with \Co{~ ET/Wafer} and read ``wafer
within ET''.  If both levels of nested factors are to be associated
with random effects then this is all you need to know.  You would use
an expression with a \Co{"/"} in the grouping factor part of the
formula for the \Co{groupedData} object.  Then the random effects
could be specified as
\begin{Example}
  random = list( ET = ~ 1, Wafer = ~ 1 )
\end{Example}
or, equivalently
\begin{Example}
  random = ~ 1 | ET/Wafer
\end{Example}

In this case, however, there would not usually be any random effects
associated with the ``experimental treatment'' or \Co{ET} factor.  The 
only random effects are at the \Co{Wafer} level.  It is necessary to
create a factor that will have unique levels for each \Co{Wafer}
within each level of \Co{ET}.  One way to do this is to assign
\begin{Example}
S> Semiconductor$Grp <-
+  getGroups( Semiconductor, form = ~ ET / Wafer, level = 2 )
S> levels( Semiconductor$Grp )  # check on the distinct levels
 [1] "1/1" "1/2" "1/3" "2/1" "2/2" "2/3" "3/1" "3/2" "3/3" "4/1"
[11] "4/2" "4/3"
\end{Example}
after which we could specify \Co{random = ~ 1 | Grp}.

\subsection{General approach}
\label{sec:generalApproach}

As a general approach to importing data into \s{} for mixed-effects
analysis you should:
\begin{itemize}
\item Create a \Co{data.frame} with one row per observation and one
  column per variable.
\item Use \Co{ordered} or \Co{as.ordered} to explicitly convert any
  ordered factors to class \Co{ordered}.
\item Use \Co{ordered} or \Co{as.ordered} to explicitly convert any
  ordered factors to class \Co{ordered}.
\item If necessary, use \Co{getGroups} to create a factor with unique
  levels from inner nested factors.
\item Specify the formula for the response, the primary covariate and
  the grouping structure to create a \Co{groupedData} object from the
  data frame.  Labels and units for the response and the primary
  covariate can also be specified at this time as can \Co{outer} and
  \Co{inner} factor expressions.
\item Plot the data.  Plot it several ways.  The use of trellis
  graphics is closely integrated with the \Co{nlme} library.  The
  trellis plots can provide invaluable insight into the structure of
  the data.  Use them.
\end{itemize}

\section{Contrasts}
\label{sec:contrasts}

When comparing estimates produced by \Co{SAS PROC MIXED} and by
\Co{lme} one must be careful to consider the contrasts that are
used to define the effects of factors.  In \textsf{SAS} a model with
an intercept and a qualitative factor is defined in terms of the
intercept and the indicator variables for all but the last level of
the factor.  The default behaviour in \s{} is to use the Helmert
contrasts for the factor.  On a balanced factor these provide a set of 
orthogonal contrasts.  In \R{} the default is the ``treatment''
contrasts which are almost the same as the SAS parameterization except 
that they drop the indicator of the first level, not the last level.

When in doubt, check which contrasts are being used with the
\textsf{contrasts} function.

To make comparisons easier, you may find it worthwhile to declare
\begin{Example}
S> options(contrasts = c(factor = "contr.SAS",
+                        ordered = "contr.poly"))
\end{Example}
at the beginning of your session.

\bibliography{Usinglme}
\end{document}

%%% Local Variables: 
%%% mode: latex
%%% TeX-master: t
%%% End: 
